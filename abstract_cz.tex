%%% A template for a simple PDF/A file like a stand-alone abstract of the thesis.

\documentclass[12pt]{report}

\usepackage[a4paper, hmargin=1in, vmargin=1in]{geometry}
\usepackage[a-2u]{pdfx}
\usepackage[utf8]{inputenc}
\usepackage[T1]{fontenc}
\usepackage{lmodern}
\usepackage{textcomp}

\begin{document}
Vrcholové pokrytí grafu \(G\) je podmnožina vrcholů, která obsahuje
alespoň jeden vrchol z~každé hrany.  Velikost nejmenšího vrcholového
pokrytí se značí \(\tau\). Pokud navíc po vrcholech z vrcholového
pokrytí chceme, aby indukovaly souvislý podgraf, pak se tato množina
nazývá souvislé vrcholové pokrytí. Velikost minimálního vrcholového
pokrytí se značí \(\tau_c\).  Rozhodovací verze obou problému jsou
NP-úplné.

K lepšímu porozumění vztahů obou grafových invariantů autoři Cardinal
and Levy zadefinovali cenu souvislosti grafu jako poměr čísel
\(\tau_c\) a \(\tau\). Není překvapující, že rozhodnout, zda cena
souvislosti pro daný graf je nejvýše rovna číslu \(t\), je NP-těžký
problém.  Myšlenku ceny souvislosti je možno zobecnit i pro jiné
grafové vlastnosti, jako například dominující množinu. Koncept ceny
souvislosti už byl zkoumán v předchozí literatuře, z čehož některé
články se soustředí na tzv. kritické grafy, což jsou grafy, kde cena
souvislosti je ostře větší než cena souvislosti každého indukovaného
podgrafu.

Tato práce obsahuje rozšířený přehled současného vědění v oblasti ceny
souvislosti.  Dále se soustředí na strukturální vlastnosti kritických
grafů a rozebírá možnou charakterizaci grafů, ve kterých cena
souvislosti za dominující množinu je nejvýše $\frac{7}{3}$ pro každý
indukovaný podgraf.
\end{document}
