%%\documentclass[tikz,border=3mm]{standalone}
%%\usetikzlibrary{shapes,snakes}
%%\usetikzlibrary{backgrounds}
%%\usetikzlibrary{calc}
%%\begin{document}
        \begin{tikzpicture}
                %%\tikzstyle{vertex}      = [circle, minimum width=4pt, draw, inner sep=0pt, fill=white]

               \begin{scope}[every node/.style={circle,thin,draw, fill=white}]
		\node (v) at (1, 1) {};
                \node (v1) at (0, 2) {};
                \node (v2) at (2, 2) {};
                \node (v3) at (2, 0) {};
		\node (v4) at (0, 0) {};

                \foreach \y in {2,3}
		{
			\foreach \z in {1,2,3}{
				\node at ($(v\y)+(\z,0)$) {};
			}
		}

		 \foreach \y in {1,4}
                {
                        \foreach \z in {1,2,3}{
                                \node at ($(v\y)-(\z,0)$) {};
                        }
                }

		\end{scope}

		\begin{scope}[on background layer][every node/.style={fill=white,circle}, every edge/.style={draw=red,very thick}]
			\draw (v2)-- ++(1,0)-- ++(1,0)-- ++(1,0);
                	\draw (v3)-- ++(1,0)-- ++(1,0)-- ++(1,0);
			\draw (v1)-- ++(-1,0)-- ++(-1,0)-- ++(-1,0);
			\draw (v4)-- ++(-1,0)-- ++(-1,0)-- ++(-1,0);

			\draw (v1)--(v2)--(v3)--(v4)--(v1);
			\foreach \x in {1,...,4}
                        	\draw (v) -- (v\x);

		\end{scope}

	\end{tikzpicture}
%\end{document}


