\chapter{Introduction}
We say that a set of vertices is a \emph{vertex cover} of a graph 
if it contains at least one endpoint from each edge. 
The size of the smallest vertex cover is called the vertex cover number.
Karp in 1972 placed the decision version of vertex cover among first 21 NP-complete problems.
Many variants of vertex cover have been studied since in literature; one of those is \emph{connected vertex cover}.
A connected vertex cover of a graph is a vertex cover that induces a single connected subgraph. 
The connected vertex cover number is the size of a minimum connected vertex cover.
This problem aside its theoretical importance has many real-world applications for example in design of wireless networks.

Let us consider the following model: the nodes in the network are connected via transmission links.
The signal decreses with transmission distance. 
To counteract this we can equip nodes with relay stations amplifying the signal.
We want to place relay stations in such a way that every transmission link is adjacent to a relay station
and any pair of relay stations has to be connected by a path containing only other relay stations.
Our budget is limited so we wish to minimize the number of relay stations used.
The solution of this problem is exactly the minimum connected vertex cover.

The mininum connected vertex cover can be also used to solve top right access point minimum length corridor problem (TRA-MLC for short.)\cite{Priyadarsini07}.
On the input is a rectangle \(F\) partioned into rectangular polygons \(P_1, \dots, P_k\).
A set \(S\) of line segments lying along the boundary of \(F\) and along boundaries of polygons is called a \emph{corridor}. 
The \emph{length of the corridor} \(S\) is a sum of lengths of all line segments included in \(S\). 
Our goal is to find a corridor with a minimum length such that line segments form a tree and contain at least one point from \(F\) and each \(P_i\).
A variant of this problem, when \(S\) has to include top-right corner of the boundary of \(F\),
is called \emph{top right access point minimum length corridor problem}.
TRA-MLC has applications in laying electrical wiring or optic-fiber cables in floor plans.

We can imagine that \(F\) represents a floor plan and \(P_i\) are individual rooms. 
We have to lay optic-fiber cables in order to grant internet access to every room. 
Such cables are expensive, so we would like the length of the used cable to be as small as possible.

A natural question to investigate is the relation between vertex cover number and connected vertex cover number.
For that purpose Cardinal and Levy define the \emph{price of connectivity} to be a ratio between the connected vertex cover number and the vertex cover number.
Deciding for a given graph whether price of connectivity is at least \(r\) is NP-hard.
The price of connectivity can be generalized to any graph problem with a meaningful connected variant.
Authors of \cite{ChiarelliHartinger18} demonstrate how to use known price of connectivity of various problems to design polynomial time algorihms.

Camby and Schaudt in \cite{CambyCardinalFioriniSchaudt14} coin the term \emph{PoC-Near-perfect graphs}.
These are graphs in which price of connectivity for vertex cover is bounded by a fixed number \(t\) for any induced subgraph.
They propose a characterization of such graphs for \(t \leq \frac{3}{2}\) in the terms of forbidden induced subgraphs.
Also they define critical graphs, which are graphs with price of connectivity stricly greater than their every induced subgraph.
The same researchers in \cite{CambySchaudt14} in a similar fashion study the price of connectivity of dominating set.

The results from \cite{Bonamy18} imply that there are no PoC-near-perfect graphs for the dominating set with price of connectivity strictly between
2 and \(\frac{7}{3}\). 
Motivated by these results one can ask if there are no PoC-near-perfect graphs with price of connectivity in the interval \((3 -\frac{2}{k-1}, 3-\frac{2}{k})\).
In this work, we show that is not true by constructioning critical graphs with the price of connectivity equal to \(3 - \frac{3}{k}\).

\section{Contents of this thesis}
In the rest if this chapter we will provide basic definitions, examine properties of vertex cover and formally define the price of connectivity.
In the \Cref{chap2} we will show applications of price of connectivity in polynomial algorithms for minimum connected vertex cover and minimum 
connected feedback vertex set from \cite{ChiarelliHartinger18}.
In the \Cref{chap3} we will closely examine price of connectivity of vertex cover. To be more specific we will discuss results by Camby and Schaudt. 
The second half of the chapter focuses on deriving new structural properties of critical graphs.
In the \Cref{chap4} we would like to refresh basic properties 
of dominating sets and present results on the price connectivity of dominating set.
Finally we will present construction of critical graphs with price of connectivity
equal to \(3 - \frac{3}{k}\).

\section{Notation}
We consider only finite undirected graphs without multiple edges and loops.
Let \(G=(V,E)\) be a graph, with \(V\) being set of vertices and \(E\) set of edges.
Let \(n = |V|\). 
The \emph{neighborhood} of a vertex \(u\) is defined as \(N(u) := \{v \in V | uv \in E\}\).
The \emph{closed neighborhood} of \(u\) is \(N[u]:=N(u) \cup \{u\}\). 
A vertex \(u\) is a \emph{cut vertex} if its removal increases the number of connected components.
Similarly, an edge \(uv\) is a \emph{bridge} if its deletion increases the number of connected components.
For a set \(S \subseteq V\) we write \(G[S]\) to denote the subgraph of \(G\) induced by \(S\), i.e. a graph with the vertex set \(S\), 
where vertices are adjacent if and only if they are adjacent in \(G\).
A vertex subset \(S\) is \emph{independent} if every pair of vertices from \(S\) is non-adjacent.
The independence number \(\alpha(G)\) is the size of a maximum independent set.

We use the following standard notation to denote some special graphs:
\begin{itemize}
	\item \(P_k\) is a path on \(k\) vertices.
	\item \(C_k\) is a cycle on \(k\) vertices.
	\item \(K_{m,n}\) is a complete bipartite graph with parts of size \(m\) and \(n\).
	\item \(K_m\) is a clique on \(m\) vertices.
	\item \(H + G\) is the union of two disjoint graphs \(H\) and \(G\). In particular \(sG\) denotes
		the union of \(s\) disjoint copies of \(G\).
\end{itemize}
In this work we will focus mainly on \(H\)-free graphs.
\begin{defn}[H-free graph]\label{def1}
	Let \(G\) and \(H\) be two graphs. We say that \(G\) is \(H\)-free if it does not contain graph \(H\) as its induced subgraph.
\end{defn}
Moreover, if \(G\) neither contains \(H_1\) nor \(H_2\) as induced subgraphs we say that \(G\) is \((H_1, H_2)\)-free.

Let us show a few examples.
First, let us prove that graph \(G\) is \(P_3\)-free if and only if every connected component of \(G\) is a clique.
If \(G\) does not contain \(P_3\) it is clear that \(G\) is a union of cliques.
For the converse suppose that there is a component \(C\) that is not a clique. Pick two non-adjacent vertices from \(C\).
The shortest path between these two vertices contains at least two edges and thus \(P_3\).

Next, we claim that complete bipartite graphs can be characterized as set of all \((K_3, \overline{P_3})\)-free graphs.
From the previous proof follows that complement of \(P_3\)-free graph is a graph, in which vertices can be partitioned into independent sets
such that edges are between any pair of vertices from different independent set.
To ensure that the number of these independent sets is two, it is enough to forbid \(K_3\) as an induced subgraph.  

Finally the class of \(P_4\)-free graphs are cographs. The proof of this is more involved and we refer to \cite{Corneil81}.

\section{Vertex cover}
\begin{defn}[Vertex cover]
	A \emph{vertex cover} of a graph \(G\) is a set of vertices including at least one endpoint of every edge.
\end{defn}
The vertex cover of the smallest size is called \emph{minimum}. The vertex cover number \(\tau(G)\) is the size of a minimum vertex cover.
Vertex cover is \emph{minimal} if its every proper subset is not vertex cover.
A minimum vertex cover is always minimal, but not every minimal vertex cover is minimum. For an illustration see Example~\ref{ex1}.

Vertex cover is closely related to independent set.
\begin{obs}
	A vertex set \(S\) is a vertex cover if and only if \(V(G) \setminus S\) is an independent set.
\end{obs}
\begin{myproof}
	If \(V(G) \setminus S\) is independent, then there are no edges with both endpoints in \(G \setminus S\) 
	and thus each edge is incident to a vertex from \(S\).
	Conversely, suppose that \(S\) is a vertex cover. By definition all edges from \(E(G)\)
	have endpoints in \(S\) so \(V(G) \setminus S\) is independent.
\end{myproof}

\begin{obs}
	A vertex cover \(S\) is minimal if and only if \(V(G) \setminus S\) is a maximal independent set.
\end{obs}
\begin{myproof}
	By the previous observation \(S\) is a vertex cover if and only \(V(G) \setminus S\) is independent.
	Suppose that \(S\) is a minimal vertex cover and \(V(G) \setminus S\) is not a maximal independent set.
	Then there is a vertex \(v\) that \((V(G) \setminus S) \cup \{v\}\) is also independent.
	However, \(S \setminus \{v\}\) is vertex cover; contradiction with minimality of \(S\).
	On the other hand assume that \(G \setminus S\) is maximal independent set and that 
	\(G\) is not minimal vertex cover. 
	Since \(S\) is not a minimal vertex cover, we can find a vertex \(v\) s.t. \(S \setminus {v}\)
	is vertex cover of \(G\). The set \((V(G) \setminus S) \cup \{v\}\) is independent
	because \(v\) does not have a neighbor in \(V(G) \setminus S\). 
	That contradicts the maximality of \(V(G) \setminus S\).
\end{myproof}
In other words we have just shown that finding the minimum vertex cover is equivalent to finding maximum independent set.
As an immediate corollary, we see that the number of vertices of \(G\) is equal to \(\tau(G) + \alpha(G)\).

\begin{figure}[h]
\centering
%%Graph with definitions for VC and CVC

        \begin{tikzpicture}

		\tikzstyle{vertex}  = [circle, minimum width=14pt, draw, inner sep=0pt, fill=white]
		%%Vertices
		\node [vertex](a) at (1,1) {$a$};
		\node [vertex](b) at ($(a) + (1, 0)$) {$b$};
		\node [vertex](c) at ($(b) + (1, 0)$) {$c$};
		\node [vertex](d) at ($(c) + (1, 0)$) {$d$};
		\node [vertex](e) at (1,0) {$e$};
		\node [vertex](f) at ($(e) + (1, 0)$) {$f$};
		\node [vertex](g) at ($(f) + (1, 0)$) {$g$};
		\node [vertex](h) at ($(g) + (1, 0)$) {$h$};
		%%Edges
		\draw (a)--(e)--(f)--(b)--(c)--(g)--(h)--(d);
		\draw (f)--(b)--(g);
		\draw (f)--(c);
	\end{tikzpicture}

\caption{}
\end{figure}
\label{vc:example}

\begin{example}\label{ex1}
	Let \(G\) be a graph from Figure \ref{vc:example}.
\begin{enumerate}
	\item Set \(\{e, b, c, h\}\) is a minimum vertex cover of \(G\): \(\tau(G) = 4\).
	\item Set \(\{a, f, g, d\}\) is a maximum indepedent set of \(G\): \(\alpha(G)=4\).
	\item Set \(\{a, f, c, g, d\}\) is a minimal vertex cover of \(G\).
\end{enumerate}
\end{example}

The size of minimum vertex cover can be bounded from below by the size of maximum matching.
It is not difficult to see that the size of the minimum vertex cover is at least the maximum number of edges in a matching.
Each vertex in the minimum vertex cover is incident with at most one edge from a matching.
For bipartite graphs these two numbers are equal by the K\"onig's theorem.
Recall that maximum matching can be found in polynomial time in any graph \(G\) by Edmonds's blossom algorithm \cite{Edmonds65}.
This famous result implies that in bipartite graphs finding a minimum vertex cover is polynomially solvable.

\section{Connected vertex cover}
Now we formally define the notion of connected vertex cover which is central term in our work.
\begin{defn}[Connected vertex cover]
	A \emph{connected vertex cover} is a vertex cover that induces a connected subgraph of \(G\).
	We denote by \(\tau_c(G)\) size of a minimum connected vertex cover.
\end{defn}

The size of minimum connected vertex cover is at least equal to the size of minimum vertex cover.
We can imagine that it is possible to find a minimum connected vertex cover by only extending minimum
vertex covers; however that is not the case. See Figure \ref{VCnotCVC} depicting a graph in which the minimum connected vertex cover
does not include any minimum vertex cover.

\begin{figure}[b]
\centering
\begin{minipage}{.45\textwidth}
        \centering
        \begin{tikzpicture}
\tikzstyle{vertex}  = [circle, minimum width=4pt, draw, inner sep=0pt, fill=white]
\node [vertex](u1) at (1, 0) {};
\node [vertex](v1) at (3, 0) {};
\node [vertex, fill=black](u2) at ($(u1) - (0, 1)$) {};
\node [vertex, fill=black](v2) at ($(v1) - (0, 1)$) {};
\node [vertex, fill=black](x1) at ($(u1)!0.5!(v1) + (0, 1)$) {};
\node [vertex](x2) at ($(u2)!0.5!(v2) + (0, -1)$) {};
 %%Edges
\draw (u1)--(u2);
\draw (u1)--(v2);
\draw (v1)--(v2);
\draw (v1)--(u2);
\draw (x1)--(u1);
\draw (x1)--(v1);
\draw (x2)--(u2);
\draw (x2)--(v2);
\end{tikzpicture}


        \caption*{}
        %%\label{fig:fex}
\end{minipage}
\begin{minipage}{.45\textwidth}
        \centering
        \begin{tikzpicture}
                \tikzstyle{vertex}  = [circle, minimum width=4pt, draw, inner sep=0pt, fill=white]
                \node [vertex, fill=black](u1) at (1, 0) {};
                \node [vertex, fill=black](v1) at (3, 0) {};
                \node [vertex, fill=black](u2) at ($(u1) - (0, 1)$) {};
                \node [vertex, fill=black](v2) at ($(v1) - (0, 1)$) {};
                \node [vertex](x1) at ($(u1)!0.5!(v1) + (0, 1)$) {};
                \node [vertex](x2) at ($(u2)!0.5!(v2) + (0, -1)$) {};
                %%Edges
                \draw (u1)--(u2);
                \draw (u1)--(v2);
                \draw (v1)--(v2);
                \draw (v1)--(u2);
		\draw (x1)--(u1);
                \draw (x1)--(v1);
                \draw (x2)--(u2);
                \draw (x2)--(v2);
 \end{tikzpicture}

        \caption*{}
        %%\label{fig:fsprime}
\end{minipage}
	\caption{A graph in which the minimum connected vertex cover contains no minimum vertex cover.
	Minimum connected vertex cover is depicted by black vertices on the right.
	Black and white vertices on the left-hand picture represent two minimum vertex covers. }
	\label{VCnotCVC}
\end{figure}

\begin{obs}
A graph \(G\) on \(n\) vertices has a connected vertex cover of size \(k\) if and
only if \(G\) contains the star \(K_{1,n-k}\) as a contraction.
\end{obs}

\begin{myproof}
	Let \(S\) be a connected vertex cover of size \(k\). 
	Contracting every edge in \(S\) transforms \(G\) into \(K_{1, n-k}\).
	If \(G\) contains \(K_{1, n-k}\) as a contraction, then
	\(G\) can be partitioned into induced connected subgraphs \(A, B_1, B_2, \dots, B_{n-k}\)
	in such a way that for each \(B_i\) there is at least one edge incident to a vertex from \(A\)
	and no edges between vertices from different \(B_i\).
	Move each vertex from \(B_i\) adjacent to vertex from \(A\) to \(A\) until all subgraphs \(B_i\)
	contain only one vertex. Such modified set \(A\) is connected and vertices from \(A\) covers all edges from \(G\).
	Notice that \(|V(A)| \leq k\). So \(A\) is now connected vertex cover of \(G\) with size at most \(k\).
\end{myproof}
By using this observation we can reduce the size of input graph while looking for a minimum vertex cover.
Edge \(uv\) can be contracted whenever both its endpoints belong to a connected vertex cover.
This idea was used in the algorithm designed in \cite{JohnsonPaesaniPaulusma20}.

Decision versions of both problems are defined as follows:

\begin{problem}
	\problemtitle{Vertex Cover (VC)}
        \probleminput{A graph \(G = (V,E)\) and a positive integer \(k\)}
        \problemquestion{Does \(G\) has a vertex cover \(S\), such that, \(|S|\leq k\)?}
\end{problem}

\begin{problem}
	\problemtitle{Connected Vertex Cover (CVC)}
	\probleminput{A graph \(G = (V,E)\) and a positive integer \(k\)}
	\problemquestion{Does \(G\) has a connected vertex cover \(S\), such that, \(|S|\leq k\)?}
\end{problem}

Finding a minimum vertex cover is NP-complete problem in planar graphs. %%Zdroje
On the other hand it becomes polynomially solvable for special classes of graphs: bipartite graphs (as we mentioned earlier), 
chordal graphs \cite{Gavril74}, series-parallel graphs \cite{TakamizawaKazuhikoNishizeki82}.

Connected vertex cover has been in 1977 introduced and closely investigated by \citet{GareyJohnson77}, where they established its NP-completeness
for planar graphs of maximal degree at most \(4\).
\citet{FernauManlove09} strengthen this result for planar bipartite graphs with degree at most 4. 
NP-completeness of a connected vertex cover was proved for 2-connected planar graphs with maximal degree 4 by \citet{PriyadarsiniHemalatha08}. 
Same result was showed by \citet{WatanabeKajitaOnaga91} even for 3-connected graphs.
Minimum connected vertex cover is polynomially solvable for chordal graphs \cite{EscoffierGourv`esMonnot10}, graphs with maximal degree at most three and trees \cite{Ueno88}.

In the brief overview from above we can notice that for bipartite graphs complexities of these two problems differ.
Interesting question to study is to find for which graph classes complexities of minimum vertex cover and connected minimum vertex cover
coincide.

Now we turn our attention to the graphs with one forbidden subgraph: \(H\)-free graphs.
\citet{Munaro17} prove that CVC is NP-complete in \(H\)-free graphs if \(H\) contains induced cycle or claw (\(K_{1,3}\)).
On the other hand VC becomes tractable if \(H\) contains a claw \cite{Sbihi80}\cite{Minty80}. 
Even among \(H\)-free graphs we can found class for which complexities of both problems are not the same. 

It remains to examine graphs without induced linear forest e.g. \(P_r\)-free graphs. 
Recently \citet{JohnsonPaesaniPaulusma20} create polynomial algorithm for CVC in \((sP_1 + P_5)\)-free graphs 
and \citet{Grzesik19} prove polynomiality of VC in \((sP_1 + P_6)\)-free graphs. 
Complexity of CVC for \(r \geq 6\) and of VC \(r \geq 7\) is still unknown.

\section{Price of connectivity} 
Different complexity results for these two problems rise the question how graph invariants \(\tau_c\) and \(\tau\) are related. 
To investigate that we define concept of the price of connectivity.
\begin{defn}[Price of connectivity]
	The \emph{price of connectivity} for a class of graphs \(\mathcal{G}\) is defined as the maximal ratio \(\frac{\tau_c(G)}{\tau(G)}\) over all graphs \(G \in \mathcal{G}\).
\end{defn}

The notion of the price of connectivity has been introduced by \citet{CardinalLevy10} as defined above. 
Also they proved that PoC is upper-bounded by \(\frac{2}{1 + \epsilon}\) in graphs with average degree \(\epsilon n\).
The price of connectivity can be generalized for any graph property for which its connected variant makes sense.
\begin{defn}[Price of connectivity for property \(\pi\)]
	For a class of graphs \(\mathcal{G}\) the \emph{price of connectivity for property \(\pi\)} is defined as \(\max_{G \in \mathcal{G}}(\frac{\pi(G)}{\pi_c(G)})\), 
	where \(\pi(G)\) denotes the size of a minimum set satisfying property \(\pi\) and similarly \(\pi_c(G)\) is the size of a minimum connected set with property \(\pi\).
\end{defn}
The price of connectivity was further studied for dominating set, feedback vertex set, face hitting set and odd cycle transversal set.
Later we will thoroughly investigate the price of connectivity of vertex cover and dominating set.
%%Doplnit citace of clanky

\section{Complexity}
Deciding whether the price of connectivity for vertex cover is at most \(t\) for given graphs is NP-hard. 
However, theorem stated below implies that is not clear whether it lies in NP. 

A complexity class \(\Theta_2^p\) contains decision problems which are computable in polynomial time by 
a deterministic Turing machine which is allowed to use NP-oracle \(O(\log(n))\) times, where \(n\) is the size of the input.
Examples of \(\Theta_2^p\)-complete problems are 
Odd-max-true-3SAT and Min-card-vertex-cover compare defined below. For more details about the class \(\Theta_2^p\) we refer to \cite{Spakowski00}.

\begin{problem}
        \problemtitle{Odd-max-true-3SAT}
	\probleminput{3-CNF formula \(F\)}
        \problemquestion{Is the maximum number of 1’s in satisfying truth assignments for \(F\) odd?}
\end{problem}

\begin{problem}
        \problemtitle{Min-card-vertex cover compare}
        \probleminput{Two graphs \(G_1 = (V_1,E_1)\) and  \(G_2 = (V_2,E_2)\)}
	\problemquestion{Is true that \(\tau(G_1) < \tau(G_2)\)?}
\end{problem}

It is easy to see that PoC problem belongs to \(\Theta_2^p\). 
We can use oracle to determine \(\tau(G)\) and \(\tau_c(G)\) by binary search.

\begin{thm}[\citet{CambyCardinalFioriniSchaudt14}]\label{compl:01}
	Given any connected graph \(G\) deciding whether \(\frac{\tau(G)}{\tau_c(G)} \leq r\)
	is \(\Theta_2^p\)-complete.
\end{thm}
Authors of \cite{CambyCardinalFioriniSchaudt14} prove this result by reduction from Min-Card-Vertex-cover compare.
