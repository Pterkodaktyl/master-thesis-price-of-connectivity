%%% A template for a simple PDF/A file like a stand-alone abstract of the thesis.

\documentclass[12pt]{report}

\usepackage[a4paper, hmargin=1in, vmargin=1in]{geometry}
\usepackage[a-2u]{pdfx}
\usepackage[utf8]{inputenc}
\usepackage[T1]{fontenc}
\usepackage{lmodern}
\usepackage{textcomp}

\begin{document}

%% Do not forget to edit abstract.xmpdata.
A vertex cover of a given graph is a vertex set including at least one
endpoint from every edge.  A vertex cover number \(\tau\) is the size
of a minimum vertex cover. If the vertices from a vertex cover are
required to induce a connected subgraph, the resulting set is called a
connected vertex cover.  The corresponding parameter \(\tau_c\) is
called a connected vertex cover number. The decision versions of both
problems are NP-complete.

To better understand a relation between these two vertex cover
numbers, Cardinal and Levy define the price of connectivity as a ratio
between \(\tau_c\) and \(\tau\). It is not surprising that determining
whether the price of connectivity of a given graph is at most \(t\) is
NP-hard. The notion of price of connectivity can be extended for more
graph properties, such as for the dominating set. The price of
connectivity has already been investigated in several papers, with
some focusing on critical graphs whose price of connectivity is
strictly greater than the price of connectivity of every induced
subgraph.

This thesis provides an overview of the current state of research into
the price of connectivity. Moreover, we focus on the structural
properties of critical graphs for the price of connectivity for vertex
cover and discuss a possible characterization of graphs in which the
price of connectivity for a dominating set is at most \(\frac{7}{3}\)
for every induced subgraph.


\end{document}
