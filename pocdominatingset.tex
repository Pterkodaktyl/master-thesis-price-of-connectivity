\chapter{Price of connectivity for dominating set}\label{chap4}

\begin{defn}
A \emph{dominating set} of a graph \(G\) is a subset of vertices \(D\) such that every vertex has a neighbor in \(D\)
or is in \(D\).
The size of the smallest dominating set is denoted by \(\gamma_c(G)\) and a dominating set of this size is called \emph{minimum}.
\end{defn}

\begin{problem}
        \problemtitle{Dominating set (DS)}
        \probleminput{A graph \(G = (V,E)\) and a positive integer \(k\)}
        \problemquestion{Does \(G\) have a dominating set \(D\), such that \(|D| \leq  k\)?}
\end{problem}

The decision version of the dominating set problem for a given graph \(G\) is a well studied NP-problem. %%reference na tu ucebnici
To get acquainted with the problem, we show its NP-completeness by a reduction from vertex cover.

\begin{thm}
	The dominating set problem is NP-complete \cite{GareyJohnson79}.
\end{thm}
\begin{myproof}
We prove a slightly stronger statement: For every graph \(G\) there
exists a graph \(G'\) such that every vertex cover of \(G\) of size $k$ corresponds
to a vertex cover in \(G'\) of size \(k+1\), and that \(G'\) has a vertex cover of size at most $c$
if and only if \(G'\) has a dominating set of size at most $c$.

We first handle isolated vertices of \(G\), which always belong to a dominating set but never belong
to a vertex cover. Starting with \(G\), we construct a new graph \(G''\) that contains edges and vertices from \(G\) plus two additional vertices
\(x\) and \(y\). Additionally, we add the edge \(xy\) and an edge \(xv\) for every vertex \(v \in V(G)\).

Observe that \(G''\) has a vertex cover of size at most \(k + 1\) iff
\(G\) has a vertex cover of size at most \(k\), and that in \(G''\)
there are no more isolated vertices.

We construct our final graph \(G'\) as follows: To start, the graph \(G'\)
contains all vertices and edges from \(G''\).  For every edge \(uv\)
from \(E(G'')\), we add a new vertex \(w\) and edges \(uw\), \(vw\).  We
have to prove that \(G''\) has a vertex cover of size at most \(k\) if
and only if \(G'\) has a dominating set of size at most \(k\).  First,
suppose that \(G''\) has a vertex cover \(S\) of size \(k\).  The set
\(S\) contains at least one endpoint from each edge, so the original
vertices from \(G''\) are dominated.  The vertices representing edges of
the original graph are adjacent to two vertices, one of which is in
\(S\).  Thus, these vertices are also dominated.

Conversely, let \(D\) be a dominating set of \(G'\).  Notice that if
\(D\) includes a vertex \(w\) added for an edge \(uv\), we can replace
it by \(u\) or \(v\); we are allowed to do this as vertex \(w\)
dominates only \(u\), \(v\) and these three vertices induce a
triangle.  Since every additional vertex is dominated, such a modified
dominating set \(D\) contains at least one endpoint from every edge.
The set \(D'\) is vertex cover of size at most \(k\) of \(G''\).
\end{myproof}

\section{Dominating set and independence}
The size of a minimum dominating set that is also independent is called the \emph{independent domination number} and denoted by \(i(G)\).
The \emph{upper domination number} \(\Gamma(G)\) is the size of the largest minimal dominating set.
From these definitions we can immediately see that \(\gamma(G) \leq i(G) \leq \alpha(G)\).

\begin{obs}\label{DS:InDo}
	An independent set \(S\) is maximal independent if and only if it is independent and dominating.
\end{obs}
\begin{myproof}
	Suppose that \(S\) is a maximal independent set. For any vertex \(v\) from \(V(G) \setminus S\) the set \(S \cup \{v\}\) is no longer independent,
	which means that every vertex in \(V(G) \setminus S\) is adjacent to at least one vertex from \(S\).
	Hence, \(S\) is a dominating set.
	
	On the contrary, let \(S\) be a dominating and independent set and suppose that it is not maximal.
	There is a vertex \(u\) from \(V(G) \setminus S\) such that \(S \cup \{u\}\) is independent.
	However, \(S\) is not dominating because \(u\) does not share any edge with vertex from \(S\). 
	This is a contradiction.
\end{myproof}

Berge in his book \cite{Berge62} observed that a vertex set is maximal independent if it is minimal dominating set.

\begin{obs}[\cite{Berge62}]
	Every maximal independent set in a graph \(G\) is a minimal dominating set of \(G\).
\end{obs}

\begin{myproof}
	Let \(S\) be a maximal independent set in \(G\). By Observation~\ref{DS:InDo} \(S\) is a dominating set.
	For the sake of contradiction suppose that \(D\) is not a minimal dominating set.
	In \(S\) there is a vertex \(v\) such that the set \(S \setminus \{v\}\) is dominating.
	If the set \(S \setminus \{v\}\) dominates vertices from \(V \setminus S \setminus \{v\}\), then at least one vertex from
	\(S\) must be adjacent to \(v\) and \(S\) is not independent, which leads to a contradiction.
\end{myproof}

A direct corollary of this observation is the following inequality which is a part of the so-called \emph{domination chain}.
For more information on the chain we refer to Chapter 3 of the book \cite{HaynesHedetniemiSlater98} written by Haynes, Hedetniemi and Slayter.
\begin{cor}
	For any graph \(G\)\:
	\[\gamma(G) \leq i(G) \leq \alpha(G) \leq \Gamma(G).\]
\end{cor}

The domination chain does not need to be strict; if there is a graph \(G\) for which every induced subgraph \(H\) satisfies \(i(H) = \gamma(G)\),
we call such graph is called \emph{domination perfect}.
The domination perfect graphs were defined by Sumner and Moore in 1979~\cite{SummerMoore79} who were inspired by a notion of perfect graphs.
% Since then, there were several attempts to characterize this class.
The characterization of the class was open for some time. Finally, in 1995, Zverovich and Zverovich~\cite{ZverovichZverovich95} found a characterization in terms of 17 forbidden induced subgraphs.
Aside from the above, many other variants of domination are also investigated. The focus will be on one of them, namely connected domination.

\section{Connected dominating set}

\begin{defn}
	A \emph{connected dominating set} of a graph \(G\) is a dominating set \(D\) such that \(G[D]\) is connected.
	The size of the smallest dominating set is denoted by \(\gamma_c(G)\) and a connected dominating set of the size \(\tau_c(G)\) is called 
	a minimum connected dominating set.
\end{defn}

A connected dominating set can exist only in connected graphs (a DS must include at least one vertex from each connected component).
Sampathkumar and Walikar, who were among the first researchers investigating connected domination, observe in \cite{SampathkumarWalikar79}
the following property of the connected domination number:

\begin{thm}[\cite{SampathkumarWalikar79}]
	Let \(G\) be a graph and \(H\) its spanning subgraph.
	Then, \(\gamma_c(G) \leq \gamma_c(H)\).
\end{thm}

Another interesting property of connected dominating sets regarding spanning trees is the following.
\begin{thm}\cite{HedetniemiLaskar84}
	Let \(G\) be a graph on \(n \geq 3\) vertices.
	Denote the maximum number of leaves over all spanning trees of \(G\) by \(\epsilon_T(G)\).
	Then, \(\gamma_c(G) = {(n - \epsilon_T(G))} \leq n-2\).
\end{thm}

\begin{myproof}
	Let \(T\) be a spanning tree with \(\epsilon_T(G)\) vertices. Since \(T\) without leaves is a connected dominating set of \(G\)
	the following inequality follows: \(\gamma_c(G) \leq {n - \epsilon_T(G)}.\)
	
	Suppose that \(D\) is a connected dominating set of \(G\). 
	The set \(D\) is connected and thus it has a spanning tree \(T'\).
	We can add all vertices from \(V(G) \setminus D\) to \(T'\) in such a way that each remaining vertex has only one edge to \(T'\).
	We created a new spanning tree \(T\) of the entire graph \(G\). The number of added vertices is at most \(\epsilon_T(G)\).
	The size of a minimum connected dominating set is at least \({n - \epsilon_T(G)}\).
\end{myproof}

Camby and Schaudt discovered in \cite{CambySchaudt16} that any minimal connected dominating set in \(P_t\)-free graph
is either isomorphic to \(P_{t-2}\), or it does not contain \(P_{t-2}\) as an induced subgraph.
This theorem was used in several papers to design polynomial algorithms in \(P_t\)-free graphs.
Specifically, Johnson, Paesani and Paulusma~\cite{JohnsonPaesaniPaulusma20} use it to construct an efficient algorithm for finding a minimum connected vertex cover in \((sP_1 + P_5\))-free graphs
and Bonomo et. al.~\cite{Bonomo18} apply it to create a polynomial algorithm determining whether a given \(P_7\)-free graph is three colorable.
Moreover, in the same paper, Camby and Schaudt show that a connected dominating set with these properties can be found in polynomial time.

\begin{thm}\label{DS:P-2freeT}
	For all \(t \geq 3\), any connected \(P_t\)-free graph has a connected dominating set whose induced subgraph is either \(P_{t-2}\)-free, 
	or isomorphic to \(P_{t-2}\).
\end{thm}

We will not show their proof in its entirety. 
Instead we will explain to a reader a proof of a weaker lemma which is used in the original proof by Camby and Schaudt.
What is more, this lemma will come in handy later in proofs of Theorems~\ref{DSpoc:2} and \ref{DSpoc:3} characterizing Near-PoC-perfect graphs for dominating set.
\begin{lemma}\label{DS:P-2free}
Let \(G\) be a connected graph that is \((P_k , C_k )\)-free for some \(k \geq 4\) and
let \(X\) be a minimal connected dominating set of \(G\). Then \(G[X]\) is \(P_{k-2}\)-free.
\end{lemma}
\begin{myproof}
	Let \(X\) be a minimum connected dominating set of \(G\).
	On the contrary suppose that \(X\) contains an induce path \(v_1, \dots, v_{k-2}\).
	Look at the set \(X \setminus \{v_1\}\).
	Because \(X\) is a minimum dominating set, \(X \setminus \{v_1\}\) is not dominating or not connected.
	In the first case in \(G\) there is a vertex \(v'_1\) adjacent only to one vertex from \(X\):\(v_1\).
	
	In the second case vertices \(v_2, \dots, v_{k-2}\) are all in one connected component. 
	Vertex \(v_1\) has a neighbor \(v'_1\) in a different connected component.
	In both cases \(v_1\) has a neighbor that is not adjacent to any from the vertices \(v_2, \dots v_{k_2}\).
	We can use the same reasoning for a vertex \(v_{k-2}\).
	Vertices \(v'_1, v_1, \dots, v_{k-2}, v'_{k-2}\) induce path or cycle on \(k\) vertices which contradicts
	the fact that \(G\) is \((P_k, C_k)\)-free.
\end{myproof}

\section{Price of connectivity}
We can define price of connectivity for dominating set in the same way as we did for vertex cover in \Cref{chap3}.
\begin{defn}
	For a graph \(G\), the price of connectivity (PoC) for dominating set is defined as the ratio \(\frac{\gamma_c(G)}{\gamma(G)}\).
\end{defn}

\begin{obs}
	The price of connectivity for any graph lies in the interval \([1, 3)\).
\end{obs}

\begin{myproof}
	Let \(D\) be a dominating set of \(G\) such that \(G[D]\) consists of \(c\) connected components.
	To make \(D\) connected it is enough to add at most \(2c - 2\) vertices. The distance between vertices in \(D\) is at most \(2\),
	since \(D\) is dominating.
	Assume that minimum dominating set of \(D\) has size \(k\).
	Using the previous observation:
	\[\frac{\gamma_c(G)}{\gamma(G)} \leq {\frac{k(2c - 2)}{k}} \leq {\frac{c(2c-2)}{c}} = {\frac{3c - 2}{c}} < 3.\]
\end{myproof}
Paths and cycles are examples of graphs attaining these bounds. Analogously, we can define PoC-perfect and PoC-near-perfect graphs for the dominating set:

\begin{defn}
	Graph \(G\) is \emph{PoC-perfect} if for every subgraph \(H\) of \(G\) it holds \(\gamma_c(H) = \gamma(H)\).
\end{defn}

\begin{defn}
	Graph \(G\) is \emph{PoC-near-perfect} with a threshold \(t \in (1, 3)\) if for every subgraph \(H\) of \(G\) this inequality holds \(\frac{\gamma_c(H)}{\gamma(H)} \leq t\).
\end{defn}
Zverovich in \cite{Zverovich03} provides characterization of PoC-perfect graphs.
\begin{thm}[\citet{Zverovich03}]\label{DSpoc:1}
The following assertions are equivalent:
\begin{enumerate}
	\item For a given graph \(G\) and every connected subgraph \(H\) it holds: \(\gamma_c(H) = \gamma(H).\)
	\item Graph \(G\) is \((P_5, C_5)\)-free.
\end{enumerate}
\end{thm}
Camby and Schaudt continue this line of research and in \cite{CambySchaudt14} they give a forbidden subgraphs characterization of PoC-near-perfect graphs 
with \(\gamma_c(H) \leq {\gamma(H) + 1}\):
\begin{thm}[\citet{CambySchaudt14}]\label{DSpoc:2}
	For every connected subgraph \(H\) of a graph \(G\) the inequality \(\gamma_c(H) \leq {\gamma(H) + 1}\) holds if and only if
	\(G\) is \((P_6, C_6)\)-free.
\end{thm}

\begin{myproof}
It is easy to see that \(\gamma(C_6) = \gamma(P_6) = 2\) and \(\gamma_c(C_6) = \gamma_c(P_6) = 4\).
These two graphs violate the inequality \(\gamma_c(H) \leq {\gamma(H) + 1}\) and therefore they cannot be induced subgraphs of \(G\).

Assume that \(G\) is \((P_6, C_6)\)-free.
Consider any connected induced subgraph \(H\) of \(G\). Observe that \(H\) is also \((P_6, C_6)\)-free.
To prove the sufficiency of our condition, we need to show that \(H\) satisfies \(\gamma_c(H) \leq {\gamma(H) + 1}\).
Let \(D\) be a minimum dominating set of \(H\) with connected components \(D_1, \dots, D_k\).
The set \(D\) must have at least two connected components, otherwise \(\gamma_c(H) = \gamma(H)\) and we are done.
Consider the smallest set of vertices \(C\) such that \(H[D \cup C]\) is connected.
Denote \(X\) a minimum connected dominating set of \(H[D \cup C]\). We know that \(\gamma_c(H) \leq |X|\).
	Define the set of indices \(I = \{i \subseteq \{1, \dots, k\}: D_i \cap X = \emptyset\}\). For each \(i \in I\) consider \(x_i \in X\) such that
\(x_i\) is adjacent to a vertex from \(D_i\). Such vertex exists because \(X\) is dominating. The vertices \(x_i\) are all from \(C\)
and they do not have to be distinct.
Define the set \(S \subseteq X\):
	\[S = \bigcup_{i \not\in I}{D_i \cap X} \cup \{x_j: j \in I\}.\]
The size of \(S\) is by definition at most
	\[ \sum_{i \not\in I}{|D_i \cap X|} + |I|\]
and that is at most \(|D|\).
A minimum dominating set has to include at least one vertex from every component.

We distinguish between two cases according to connectivity of \(H[S]\).

Suppose that \(H[S]\) is connected. Then \(H[D \cup \{x_j : j \in I\}]\) is connected and \(C\) contains only vertices \(x_j\).
Recall that \(C\) is a minimal set with this property.
In this case \(X = S\) and this yields:
	\[\gamma_c(H) \leq |X| = |S| \leq {\sum_{i \not\in I}{|D_i \cap X|} + |I|} \leq |D| = \gamma(H).\]

\(H[S]\) is not connected.
By Lemma~\ref{DS:P-2free} \(H[X]\) is \(P_4\)-free. 
It is known that \(P_4\)-free graphs on at least two vertices are either disconnected or
their complement is disconnected \cite{Corneil81}
Specifically, it means that the graph \(\overline{H[S]}\) is connected.
Since \(H[X]\) is connected, the graph \(\overline{H[X]}\) is disconnected and it contains the set \(S\) in one connected component and a nonempty set of vertices \(Y\)
which are not adjacent to any vertex from \(S\).
What can we say about structure of graph \(H[X]\)? 
In \(X\), each vertex in \(Y\) is adjacent to every vertex from \(S\).
Thus, we can pick any vertex from \(y \in Y\) such that \(H[D \cup \{x_j: j \in I\} \cup \{y\}]\) is connected.
So \(|C| = \{x_j: j \in I\} \cup \{y\}\) and \(X = S \cup \{y\}\).
This gives us: 
	\[\gamma_c(H) \leq {|X| + 1} = {|S| + 1} \leq {\sum_{i \not\in I}{|D_i \cap X|} + |I| + 1} \leq {|D| + 1} = \gamma(H) + 1.\]
This concludes the proof.
\end{myproof}
To see that this bound is sharp, we consider the graphs \(F_k\) which are constructed from \(K_{1, k}\)
by subdividing every edge once. See Figure \ref{pic:fks} for an illustration.

\begin{figure}
        \centering
        \begin{minipage}{.5\textwidth}
                \centering
                \input{pictures/DSpocfksDS}
                \caption*{}
        \end{minipage}%
        \begin{minipage}{.5\textwidth}
                \centering
                \begin{tikzpicture}%%minCDS
                \tikzstyle{vertex}  = [circle, minimum width=4pt, draw, inner sep=0pt, fill=white]
                \node [vertex, fill=black](x) at (3,3) {};
                \foreach \i in {1,...,4}
                {
                        \node [vertex](a\i) at ($(x)+(({45 + \i*90}:2cm)$) {};
                        \node [vertex, fill=black](b\i) at ($(a\i)!0.5!(x)$) {};
                        \draw (b\i) -- (x);
                        \draw (a\i) -- (x);
                }
\end{tikzpicture}

                \caption*{}
        \end{minipage}
	\caption{Graph \(F_k\) for \(k=4\)
	with minimum dominating set and minimum connected dominating set indicated by black vertices.}
        \label{pic:fks}
\end{figure}

Camby and Schaudt~\cite{CambySchaudt14} also attempt to characterize PoC-near-perfect graphs with a parameter \(t=2\),
presenting the following theorem:

\begin{thm}[\citet{CambySchaudt14}]\label{DSpoc3}
For every \((P_8 , C_8)\)-free graph \(G\), it holds that
	\(\gamma_c(G) \leq 2 \gamma(G).\)
\end{thm}
The same authors in \cite{CambySchaudt14} and in \cite{Camby19} proposed a conjecture how to characterize all PoC-near-perfect graphs
with parameter \(t = 2\). Recently Bonamy et al. in \cite{Bonamy18} prove that their conjecture was correct:
\begin{thm}[\citet{Bonamy18}]\label{DSpoc:3}
The following assertions are equivalent:
\begin{enumerate}
	\item For a graph \(G\) and every induced subgraph \(H\) it holds: \(\gamma_c(H) \leq 2\gamma(H).\)
	\item The graph \(G\) is \((C_9, P_9, F)\)-free.
\end{enumerate}
\end{thm}

\begin{figure}
	\centering
        \input{pictures/graph_f}
	\caption{Graph \(F\) from Theorem~\ref{DSpoc:3}.}
	\label{DS:graphF}
\end{figure}
The remaining question is how to characterize PoC-near-perfect graphs for different values of parameter \(t\).
The construction of forbidden subgraphs from the previous theorems can be generalized in the following way:
\begin{defn}\label{DS:tk}
	Class of graphs \(\mathcal{T}_k\) consist of all graphs created as follows:
	\begin{itemize}
		\item Start with any tree on \(k\) vertices. Denote these nodes \(v_1, v_2, \dots, v_k\).
		\item Subdivide each edge twice.
		\item To each original leaf, we attach a new vertex (which becomes the new leaf).
		\item The vertices \(N(v_i)\) may be connected by new edges if \(v_i\) is not adjacent to a leaf. 
                  If at any point \(N(v_i)\) induces a connected graph, we add a new leaf to \(v_i\)
                  and we do not add any more edges into \(N(v_i)\).
		\item Finally, we may add a set $M$ of edges among leaves if $M$ is a (not necessarily maximal)
                  matching.
	\end{itemize}
\end{defn}

\begin{obs}\label{ds:7}
	For any graph \(G_k\) from \(\mathcal{T}_k\), the domination number \(\gamma\) is equal to \(k\)
	and \(\gamma_c(G_k) = 3k - 2\).
\end{obs}
\begin{myproof}
Observe that vertices \(v_i\) dominate all vertices from \(V(G_k)\), and so \(\gamma \leq k\).
Since the closed neighborhoods of individual vertices \(v_i\) are disjoint and any dominating set must
contain at least one vertex from each \(N[v_i]\), we have \(\gamma \geq k\).

The size of a minimum connected dominating set is at most \(3k-2\). 
Let us take a minimum dominating set that contains all vertices \(v_i\).
To make this dominating set connected, we need to add all vertices except leaves.
Number of edges in the initial tree is \(k - 1\), in the second step we added two vertices for each edge.
The total sum is \(k + (2k - 1) = 3k - 2\).

To finish the proof it remains to show that the size of a minimum connected dominating set is at least \(3k - 2\).
We consider a minimum connected dominating set \(D\).
If \(D\) contains all vertices \(v_i\) from the discussion above it follows that the size of \(D\) is at least \(3k-2\).

Assume that a vertex \(v_j\) is not in \(D\).
Moreover, we  suppose that \(v_j\) is adjacent to a leaf \(l_1\) and a vertex \(x_1\).
The situation is the following. 
The set \(D\) dominates \(l_1\); thus, there is a leaf \(l_2\) in \(D\) adjacent to \(l_1\).
From the construction of \(G_k\) we can see that all leaves are adjacent to the vertices \(v_i\).
In particular \(l_2\) is adjacent to a vertex \(v_m\). For an illustration see Figure~\ref{DS:DSTk}.
At least one vertex from a pair \(x_1\), \(l_1\) must be in \(D\), otherwise \(v_j\) is not dominated by \(D\).
Vertices \(l_2\), \(v_m\) and \(x_2\) are in \(D\).
Since edges on leaves induce a matching we can modify \(D\) in such a way that it includes \(v_j\)
and has at most the same size. Namely we \(D\) vertices \(v_j, x_1\) and remove \(l_1\).

Suppose that vertex \(v_j\) is not adjacent to any leaf.
In that case \(N(v_j)\) does not induce a connected subgraph.
Vertex \(v_j\) is dominated by a vertex \(w\). 
Denote by \(C_1\) a connected component of \(N(v_j)\).
In the neighborhood of \(v_j\) there is another connected component \(C_2\).
Denote by \(w'\) a vertex from \(D\) that has a neighbor in \(C_2\).
Such vertex exists because vertices from \(C_2\) needs to be dominated by \(D\).
Let \(w''\) be a common neighbor of vertices \(v_j\) and \(w'\).
Since subgraph \(H[D]\) is connected, it contains a path \(P\) from \(w\) to \(w'\).
The path \(P\) must contain an edge between two leaves \(l_1\) and \(l_2\).
We remove \(l_1\) and \(l_2\) from \(D\) and add vertices \(v_j\), \(w''\) instead.
Notice that such modified \(D\) is still a connected dominating set and
the size is at most the same as before.
For reference see Figure \ref{DS:DSTk}.
\end{myproof}

\begin{figure}
        \centering
	\begin{minipage}{.5\textwidth}
                \centering
                        \begin{tikzpicture}

		\tikzstyle{vertex}  = [circle, minimum width=4pt, draw, inner sep=0pt, fill=white]
		\node [vertex, label={left:$l_1$}](l1) at (0,0) {};
		\node [vertex, fill=red, label={right:$l_2$}](l2) at (1,0) {};
		\node [vertex, fill=red, label={right:$v_m$}](vm) at ($(l2) + (-45:1cm)$) {};
		\node [vertex, label={left:$v_j$}](vj) at ($(l1) + (135:1cm)$) {};
		\node [vertex, label={above:$x_1$}](x1) at ($(l2) + (95:1cm)$) {};
		\node [vertex, fill=red, label={above:$x_2$}](x2) at ($(x1) + (30:1cm)$) {};
		\draw (vm) -- (l2) -- (l1) -- (vj) -- (x1)--(x2);

	\end{tikzpicture}


		\caption*{}
        \end{minipage}%
	\begin{minipage}{.5\textwidth}
                \centering
                \begin{tikzpicture}

                \tikzstyle{vertex}  = [circle, minimum width=4pt, draw, inner sep=0pt, fill=white]
                \node [vertex, fill=red, label={left:$l_1$}](l1) at (0,0) {};
                \node [vertex, fill=red, label={right:$l_2$}](l2) at (1,0) {};
                \node [vertex, fill=red, label={left:$w$}](w) at ($(l1) + (110:1cm)$) {};
                \node [vertex, label={left:$v_j$}](vj) at ($(w) + (30:1cm)$) {};
                \node [vertex, label={above:$w^{''}$}](w1) at ($(vj) + (30:1cm)$) {};
                \node [vertex, fill=red, label={above:$w^{'}$}](w2) at ($(w1) + (30:1cm)$) {};
                \draw (l2) -- (l1);
		\draw  (w) -- (vj) -- (w1)--(w2);
		\draw [dashed] (w) -- (l1);
		\draw [dashed] (w2) -- (l2);
\end{tikzpicture}

                \caption*{}
        \end{minipage}%

	\caption{Illustration to the proof of Observation~\ref{ds:7}. Red vertices are in a connected dominating set \(D\)}
        \label{DS:DSTk}
\end{figure}

A direct corollary of this observation is that for any graph \(G_k\) from \(\mathcal{T}_k\) the ratio
\(\frac{\gamma_c(G)}{\gamma(G)}\) is at most \(3 - \frac{2}{k}\).
\begin{conj}[\citet{Bonamy18personal}]\label{DS:conj}
Let \(G\) be a graph.
For each subgraph \(H\) of \(G\) it holds: \({\frac{\gamma_c(H)}{\gamma(H)}} \leq {3 - \frac{2}{k - 1}}\)
if and only if \(G\) is \(\mathcal{T}_k\)-free.
\end{conj}
Let us look at graphs from \(\mathcal{T}_k\) for various values of \(k\).
\begin{itemize}
	\item \(k = 2.\)
		The class \(\mathcal{T}_2\) contains only graphs \(C_6\) and \(P_6\) which are forbidden subgraphs from Theorem~\ref{DSpoc:2}.
	\item \(k = 3.\)
		The class \(\mathcal{T}_3\) includes \(C_9\), \(P_9\), \(F\).
		Their PoC is \(\frac{7}{3}\) and they are forbidden subgraphs for graphs with PoC at most \(2\) from Theorem~\ref{DSpoc3}.
	\item \(k = 4.\)
		Some examples of graphs from  \(\mathcal{T}_4\) are \(P_{12}\), \(C_{12}\) and graphs derived from a star on \(4\) vertices (see Figure~\ref{pic:T4}).
\end{itemize}
\begin{figure}
        \centering
        \begin{minipage}{.5\textwidth}
                \centering
                \begin{tikzpicture}%%minDS

                \tikzstyle{vertex}  = [circle, minimum width=4pt, draw, inner sep=0pt, fill=white]
                \node [vertex, fill=black](v1) at (2,4) {};
                \node [vertex, fill=black](v2) at ($(v1)+(-30:3cm)$) {};
                \node [vertex, fill=black](v3) at ($(v1)+(-90:3cm)$) {};
                \node [vertex, fill=black](v4) at ($(v1)+(210:3cm)$) {};

                \foreach \i in {2,...,4}{
                        \node [vertex](x\i) at ($(v1)!0.33!(v\i)$) {};
                        \node [vertex](y\i) at ($(v1)!0.66!(v\i)$) {};
                        \node [vertex](l\i) at ($(v\i)+(0,-1)$){};
                        \draw (v1)--(x\i)--(y\i)--(v\i)--(l\i);
                }
\end{tikzpicture}


                \caption*{}
        \end{minipage}%
        \begin{minipage}{.5\textwidth}
                \centering
                \input{pictures/DSpocT4b}
                \caption*{}
        \end{minipage}
        \caption{An example of graphs from \(\mathcal{T}_4\)
        with vertices \(v_i\) indicated by black vertices.}
        \label{pic:T4}
\end{figure}

Note that not all graphs from \(\mathcal{T}_i\) are necessary critical. 
The construction from \Cref{DS:tk} allows the existence of subgraphs with the PoC equal or greater than \(3 - \frac{3}{k}\), namely 
an addition of edges to the neighborhood of \(v_i\) together with edges between leaves can create long induced paths.
This fact does not contradict our conjecture, it only means that our characterization is not minimal.

Unfortunately we will show that \Cref{DS:conj} is not true even for \(k = 3\).
To establish that we will show construction of critical graphs with PoC greater than \(3 - \frac{3}{k-1}\) without forbidden subgraphs from \(\mathcal{T}_i\).
\begin{thm}\label{DSpoc:4}
Let \(G_d\) be a graph with \(\gamma(G_d) = d\) constructed like this:
\begin{itemize}
	\item Start with a star on \(k\) vertices with central vertex \(v\).
	\item Add edges such that \(N(v)\) induces \(K_{\ceil{\frac{d-1}{2}}, \floor{\frac{d-1}{2}}}\). 
	\item To every vertex from the bipartite graph add a path on three vertices.
\end{itemize}
	Then for \(d = 5\) and \(d > 6\), the graph \(G_d\) is critical and we have \(\frac{\gamma_c(G_d)}{\gamma(G_d)} = {3 - \frac{3}{d}}\).
\end{thm}

\begin{figure}
	 \centering
         %%\documentclass[tikz,border=3mm]{standalone}
%%\usetikzlibrary{shapes,snakes}
%%\usetikzlibrary{backgrounds}
%%\usetikzlibrary{calc}
%%\begin{document}
        \begin{tikzpicture}
                %%\tikzstyle{vertex}      = [circle, minimum width=4pt, draw, inner sep=0pt, fill=white]

               \begin{scope}[every node/.style={circle,thin,draw, fill=white}]
		\node (v) at (1, 1) {};
                \node (v1) at (0, 2) {};
                \node (v2) at (2, 2) {};
                \node (v3) at (2, 0) {};
		\node (v4) at (0, 0) {};

                \foreach \y in {2,3}
		{
			\foreach \z in {1,2,3}{
				\node at ($(v\y)+(\z,0)$) {};
			}
		}

		 \foreach \y in {1,4}
                {
                        \foreach \z in {1,2,3}{
                                \node at ($(v\y)-(\z,0)$) {};
                        }
                }

		\end{scope}

		\begin{scope}[on background layer][every node/.style={fill=white,circle}, every edge/.style={draw=red,very thick}]
			\draw (v2)-- ++(1,0)-- ++(1,0)-- ++(1,0);
                	\draw (v3)-- ++(1,0)-- ++(1,0)-- ++(1,0);
			\draw (v1)-- ++(-1,0)-- ++(-1,0)-- ++(-1,0);
			\draw (v4)-- ++(-1,0)-- ++(-1,0)-- ++(-1,0);

			\draw (v1)--(v2)--(v3)--(v4)--(v1);
			\foreach \x in {1,...,4}
                        	\draw (v) -- (v\x);

		\end{scope}

	\end{tikzpicture}
%\end{document}



	 \caption{An example of a graph from Theorem~\ref{DSpoc:4} for $d = 5$.}
\end{figure}

\begin{proof}
	The size of a minimum connected dominating set of graph \(G_d\) is \(3d - 3\). It must include two vertices from each \(P_3\) and the whole bipartite subgraph.
	The minimum dominating set contains one vertex from each \(P_3\) and vertex \(v\).
	
	It remains to prove that \(G_d\) is a critical graph.
	In other words, every proper subgraph \(H\) must have a strictly smaller PoC than \(G_d\). 
	Due to relatively simple structure of \(G_d\) we analyze all possibilities of the subgraphs.
	For the rest of the proof we assume that the subgraph \(H\) is connected; the price of connectivity
        is defined only for connected graphs.
	Before diving deeper into the case analysis, we wish introduce some notation.
	
	Vertices of graph \(G_d\) can be divided into the following sets:
	\begin{itemize}
		\item \(P:=\) the set of vertices from \(P_3\) added in the last step of the construction.
		\item \(B_1, B_2 :=\) the sets of vertices from each part of the bipartite graph from the second step.
		\item \(v :=\) the central vertex of \(G_k\).
	\end{itemize}
	The number of induced paths with at least two vertices in \(P \cap V(H)\) is denoted by \(l\),
	the number of paths with one vertex in \(P \cap V(H)\) is denoted by \(s\).

	We will distinguish between several cases according to the structure of the subgraph \(H\).
	\begin{enumerate}
		\item [Case 1.] \(H \cong P_9.\)

		The price of connectivity of \(P_9\) is equal to \(\frac{7}{3}\).
        	To ensure criticality of \(G_d\), we need to set the value \(d\) such that the PoC of \(P_9\) is smaller than the PoC of \(G_d\).
        	\[\frac{\gamma_c(G_d)}{\gamma(G_d)} > \frac{7}{3}.\] 
        	This inequality implies that \(d\) must be greater than \(4\).
		We observe that \(P_9\) is the longest possible induced path in \(G_d\).

		\item [Case 2.] \(V(H) \cap P = \emptyset.\)
			
		In this case \(\frac{\gamma_c(H)}{\gamma(H)} = 1\).
		Vertices from the bipartite subgraph are dominated by either the vertex \(v\), or at most two adjacent vertices.
	\end{enumerate}
		
		The set of vertices \(H \cap P\) is nonempty. What does a minimum dominating set of \(H\) look like?
		In any dominating set, there must be at least one vertex from \(P\) for each long path and at least one vertex from \(B\) to dominate the short path.
		The minimum connected dominating set contains at most two vertices from \(P\) and one vertex from \(B\) for each path in \(H \cap P\).
		We distinguish between the following cases according to the number of short paths and the presence of vertex \(v\) in \(H\).
	\begin{enumerate}
		\item[Case 3.] \(s = 0\) and \(v \in H\).
                Every long path and the vertex \(v\) needs to be dominated, and so the size of a minimum dominating set of \(H\) is at least \(l + 1\).
                Suppose that the vertices from a minimum CDS of \(H\) (which dominate the vertices on the long paths from \(P \cap H\))
                include some from both parts \(B_1\) and \(B_2\).
		Thus \(\gamma_c(H) \leq 3l\).

		\[\frac{\gamma_c(H)}{\gamma(H)} \leq {\frac{3l}{l + 1}} = {3 - \frac{3}{l + 1}}.\]\label{case3s=0a}

		The last expression is increasing in \(l\) and \(l \leq d + 1\), so it is at most \(3 - \frac{3}{d}\).
		For smaller values of \(l\) the PoC of \(H\) is strictly smaller than \(3 - \frac{3}{d}\).
		When \(l\) is equal to \(d - 1\), at least one long path has one vertex not in \(H\), otherwise \(H \cong G\).
		Thus, we can improve the bound to:
		\[\frac{3l - 1}{l + 1} \leq {3 - \frac{4}{d}} < {3 - \frac{3}{d}}.\]\label{case3s=0b}

		In case that vertices from \(B\) adjacent to vertices from \(P \cap H\) are from the same part \(B_i\),
		the CDS contains one additional vertex that connects vertices from \(B_i\).
		Thus,  \(\gamma_c(H) \leq 3l + 1\) and we have:
		
		\[\frac{\gamma_c(H)}{\gamma(H)} \leq {\frac{3l + 1}{l + 1}} = {3 - \frac{2}{l + 1}}.\]	
		The number of paths with vertices from \(V(H) \cap P\)\ is at most \(\ceil{\frac{d - 1}{2}}\), which we plug in:
		\[{3 - \frac{2}{l + 1}} \leq {3 - \frac{2}{\frac{d + 2}{2}}} = {3 - {\frac{4}{d + 2}}} < 3 - \frac{3}{d}.\]

		The rightmost inequality is satisfied for \(d > 6\).
		This bound is not sharp for odd \(d\).
		The number of long paths is at most \(\frac{d-1}{2}\). 
		Using this bound yields the same result as before for \(d > 3\).
	
		\item [Case 4.] \(s \geq 2\) and \(v \in H\).
		
		The size of the minimum dominating set is at least \(l + s\).
		In the same manner as before, we distinguish several cases according to the size of a minimum connected dominating set.
			\begin{itemize}
				\item \(\gamma_c(H) \leq {3l + s}\). We compute:

				\[\frac{\gamma_c(H)}{\gamma(H)} \leq {\frac{3l + s}{l + s}} = {3 - \frac{2s}{l + s}} \leq {3 - \frac{4}{d - 1}} < {3 - \frac{3}{d}}.\]
				We have used that \(s\) is at least two.
			
				\item \(\gamma_c(H) \leq {3l + s + 1}\). In this situation all paths are adjacent to vertices from the same part of \(B\):
				\[\frac{\gamma_c(H)}{\gamma(H)} \leq {\frac{3l + s + 1}{l + s}} = {3 - \frac{2s - 1}{l + s}} \leq {3 - \frac{3}{l + s}}.\]

				The number of paths is at most \(\frac{d}{2}\).
				\[{3 - \frac{3}{l + s}} \leq {3 - \frac{6}{d}} < {3 - \frac{3}{d}}.\]
			\end{itemize}

		\item  [Case 5.] \(s = 1\) and \(v \in H\). We distinguish three subcases:
		
			\begin{itemize}
			\item All long paths on vertices \(V(H) \cap P\) are adjacent to one part \(B_i\).
				
			For a minimum connected dominating set, the inequality \(\gamma_c(H) \leq 3l + 2\) holds.
			By assuming \(s = 1\), a minimum connected dominating set includes a vertex from \(B\) dominating the short path. 
			Aside from this vertex, a minimum CDS also contains at most 3 vertices from each long path and at most one vertex from \(B\). 
			In particular, if one extra vertex from \(B\) is needed, then all paths on vertices \(P \cap V(H)\) are adjacent to vertices either from \(B_1\), or \(B_2\).
                        We calculate:

			\[{\frac{\gamma_c(H)}{\gamma(H)}} \leq {\frac{3l + 1}{l + 1}} = {3 - \frac{4}{d}} \leq {3 - \frac{3}{d}}.\]
			The last inequality holds for \(d > 0\).
			
			\item \(P \cap H\) induce two \(P_3\) adjacent to different parts of \(B\), or all paths \(H[V(H) \cap P]\) are adjacent to \(B_i\).
		
			This assumption implies that the size of minimum dominating set is at least \(l + 2\).
			From our estimates it follows that
			\[{\frac{\gamma_c(H)}{\gamma(H)}} \leq {\frac{3l + 2}{l + 2}} = {3 - \frac{4}{d}} < {3 - \frac{3}{d}}.\]

			
			\item The paths of length 3 with vertices from \(P \cap H\) are adjacent to one part of \(B\).
			The size of a minimum connected dominating can be expressed by:
				\[\gamma_c(H) = {3l_3 + 2l_2 + 1},\]
				where \(l_3\) is the number of \(P_3\) and \(l_2\) the number of \(P_2\) in \(H[V(H) \cap P]\).
			By our assumption, \(l_3\) is at most \(\ceil{\frac{d - 1}{2}} \leq {\frac{d}{2}}\).
			Similarly \(l_2 \leq \frac{d-1}{2}\).
				\[\gamma_c(H) = {3\frac{d}{2} + 2\frac{d-1}{2} - 1} = {\frac{5}{2}d - 2}\].
			
			Vertex dominating short path cannot be counted twice. %%Ukazat graf s timhle odhadem
			Let us bound price of connectivity of \(H\)
				\[\frac{\frac{5}{2}d - 2}{l_3 + l_2 + 1} \leq {\frac{\frac {5}{2}d - 2}{d-1}} = 2 - \frac{d}{2(d-1)}.\]
			The rightmost expression is strictly less than \(3-\frac{3}{d}\) for \(d > 7\).
			For even \(d\) bound of \(\gamma_c(H)\) can be improved because \(l_3 \leq (d-1)/2\). %\frac{d-1}{2}.
			Using this bound instead yields the same result for \(d > 3\).
			\end{itemize}
	\end{enumerate}

        \goodbreak
	\begin{enumerate}
	\item [Case 6.]\(s = 0\) and \(v \not\in H\). We go through three subcases once more:
		\begin{itemize}
			\item The graph \(H[V(H) \cap P]\) contains at least one \(P_3\) and \(\gamma_c(H) \leq 3l\).
				Due to the \(P_3\) in \(H[V(H) \cap P]\), the size of a dominating set is at least \(l + 1\).
			\[\frac{\gamma_c(H)}{\gamma(H)} \leq {\frac{3l}{l + 1}} = 3 - \frac{3}{l + 1}\]
		
			For \(l < d - 1\), the rightmost expression is strictly less than price of connectivity  of \(H\).
			In the case when \(l = d\) and at least one long path is shorter than 3, we are done
			by Case 3.

			Assume that each path from \(H[V(H) \cap P]\) is \(P_3\) and \(l = d-1\).
			If this is the case then observe that size of minimum dominating set is equal to \(l + 2\).
				\[\frac{\gamma_c(H)}{\gamma(H)} \leq {\frac{3l}{l + 2}} = 3 - \frac{6}{d + 1} < 3-\frac{3}{d}\]
			The last inequality holds for all \(d\).	

			\item Paths from \(H[V(H) \cap P]\) are adjacent to vertices from one part of \(B\).
				
			\[\frac{\gamma_c(H)}{\gamma(H)} \leq {\frac{3l + 1}{l + 1}} = 3 - \frac{2}{l + 1} 
			\leq {3 - \frac{2}{\frac{d + 2}{2}}} = 3 - \frac{4}{d + 2} < 3 - \frac{3}{d}.\]

			Since all paths are adjacent to vertices from one part of \(B\) the number of long paths is at most 
			\(\frac{d}{2}\).
			The last inequality holds for \(d > 6\).
			For odd \(d\) the number \(l\) is at most \(\frac{d - 1}{2}\).
			Using this bound instead we derive the same result for \(d > 3\).

			\item All paths in \(H[V(H) \cap P]\) have length 2.

			\[\frac{\gamma_c(H)}{\gamma(H)} \leq {\frac{2l + 1}{l}} = 2 - \frac{1}{l}\]
			Inequality \(2 - \frac{1}{l} < 3 - \frac{3}{d}\) holds for \(d > 4\).
		\end{itemize}
	
	\item [Case 7.]\(s = 1\) and \(v \not\in H\).
		
		Follows the same proof from Case 5.

	\item [Case 8.] \(s \geq 2\) and \(v \not\in H\).
		
		This situation is identical to Case 4.
\end{enumerate}
\end{proof}
The price of connectivity of graphs from \(\mathcal{T}_4\) is \(\frac{5}{2}\).
We proved that \(G_5\) is critical; thus it cannot contain
induced subgraphs with PoC greater than \(\frac{12}{5}\). In particular it cannot contain graphs with PoC equal to \(\frac{5}{2}\).
The conjecture for \(k=4\) claims that PoC-near-perfect graphs for \(t=\frac{7}{3}\) are exactly \(\mathcal{T}_4\)-free graphs.
However, \(G_5\) is a graph with price of connectivity greater than \(\frac{7}{3}\).
Hence, we found an counterexample.

In general we can derive the following corollary.
\begin{cor}
	For \(k > 5\) graphs \(G_{k + 1}\) from \Cref{DSpoc:4} are \(\mathcal{T}_{k}\)-free.
\end{cor}
\begin{myproof}
	From \Cref{DSpoc:4} graphs \(G_{k + 1}\) are critical with price of connectivity equal to \(3 - \frac{3}{k+1}\).
	By \Cref{ds:7} the price of connectivity of graphs from \(\mathcal{T}_{k}\) is equal to \(3 - \frac{2}{k}\).
	The inequality \[{3 - \frac{3}{k+1}} < {3 - \frac{2}{k}}\]
	holds for \(k > 2\). 
	Thus, graphs from \(\mathcal{T}_{k}\) cannot be induced subgraphs of \(G_{k + 1}\).
\end{myproof}

\section{Conclusions}
\Cref{chap3} discussed the characterization of PoC-near-perfect graphs for vertex cover for \(t = 1, \frac{4}{3}, \frac{3}{2}\)
as given by Camby and Schaudt. The same authors propose a possible characterization of the PoC-near-perfect graphs for \(t \leq \frac{5}{3}\). 
One possible direction of future research may be try to prove their conjecture and find a list of forbidden subgraphs for different values of \(t\).
% We also investigate the structural properties of PoC-critical graphs. 
In the same chapter, we have discussed the structural properties of PoC-cricital graphs; namely we observe that vertices from any minimum vertex cover have an independent neighborhood.
We derive a similar result for a pair of adjacent vertices from a minimum vertex cover in \Cref{VC:morecomp}.
The remaining question is whether the same property holds even for non-adjacent pair of vertices \(x, y\).

In \Cref{chap4} we show critical graphs with the price of connectivity for a dominating set equal to \(3-\frac{3}{k}\).
From the results listed in \Cref{chap4} about PoC-near-perfect graphs that in the interval \((1, 3)\), there exist rational values of the price of connectivity that
cannot be attained by any PoC-near-perfect graph. 
Interesting research problem to examine further could be to specify for which values of the price of connectivity 
PoC-near-perfect graphs exist. 

Notice that all PoC-near-perfect graphs do not contain long paths. 
Their study may be helpful in a better understanding to the class of \(P_r\)-graphs.
Recall that one of the few general characterizations of \(P_r\)-graphs (Theorem~\ref{DS:P-2freeT})
was established during an investigation of PoC-near-perfect graphs.
